\begin{DoxyAuthor}{Author}
Sune Lauth Gadegaard 
\end{DoxyAuthor}
\begin{DoxyVersion}{Version}
1.\+0.\+0
\end{DoxyVersion}
\hypertarget{index_License}{}\section{License}\label{index_License}
Copyright 2015, Sune Lauth Gadegaard. This program is free software\+: you can redistribute it and/or modify it under the terms of the G\+NU General Public License as published by the Free Software Foundation, either version 3 of the License, or (at your option) any later version.

This program is distributed in the hope that it will be useful, but W\+I\+T\+H\+O\+UT A\+NY W\+A\+R\+R\+A\+N\+TY; without even the implied warranty of M\+E\+R\+C\+H\+A\+N\+T\+A\+B\+I\+L\+I\+TY or F\+I\+T\+N\+E\+SS F\+OR A P\+A\+R\+T\+I\+C\+U\+L\+AR P\+U\+R\+P\+O\+SE. See the G\+NU General Public License for more details.

You should have received a copy of the G\+NU General Public License along with this program. If not, see \href{http://www.gnu.org/licenses/}{\tt http\+://www.\+gnu.\+org/licenses/}.

If you use the software in any academic work, please make a reference to

``\+An appropriate reference should go here!\textquotesingle{}\textquotesingle{}  
\section{Description}

This software provides a "converter" that given appropritate input, generates a result file compatible witht format used on \href{https://github.com/MCDMSociety}{MOrepo}.
The software basically consists of a single class "converter". After constructing an object of the class, the simple "set" methods should be used to set the appropriate
entries, and finally, after setting atleast all required entries, the \texttt{createResultsFile (...)} function should be called.
\hypertarget{index_Compiling}{}\section{Compiling}\label{index_Compiling}
The codes were compiled using the Visual Studio 2015 compiler on a Windows 10 machine. The following flags were used\+: /\+W3 /\+Ox /std\+:\mbox{[}c++14$\vert$c++latest\mbox{]}

 
  \section{Example of usage}
  This section contains an example showing how the \texttt{converter} class can be used to generate a MOrepo compatible json file. In this example we assume the instance
  is named "instance1", that the contribution is "Foo et al. 2017", and that the class dataReader has the appropriate get methods.

  \begin{lstlisting}
      // main.cpp
    #include"converter.h"
    #include"dataReader.h" // Assume you have written this yourself
    #include<vector>
      int main(int argc, char** argv){
          try{

            std::vector<std::string> allArgs(argv, argv + argc); // retrieve arguments

              if ( argc < 2 )
            {
                throw std::runtime_error ( "Two few arguments. No input file was specified" );
            }
            else if ( argc < 3 )
            {
                outputFile = "./results.json";
            }
            else
            {
                outputFile = allArgs[2];
            }
            inputFile = allArgs[1];

            dataReader dr = dataReader ( inputFile ); //Read your own result file

            converter conv = converter(); // Create a converter object
            conv.setVersion ( "1.0" );
            conv.setInstanceName ( "instance1" );
            conv.setContributionName ( "Foo et al. 2017" );
            conv.setObjectives ( dr.getNumberOfObjectives ( ) );
            conv.setObjectiveTypes ( dr.getObjectiveTypes ( ) );
            conv.setOptimal ( dr.isItOptimal ( ) );
            conv.setCardinality ( dr.getCardinalityOfFrontier ( ) );
            conv.setPoints ( dr.getPoints ( ) , dr.getPointTypes ( ) );
            conv.setValidity ( dr.getValidityOfSolution ( ) );
            conv.createResultsFile ( outputFile );

            return 0;
        }
        catch ( std::runtime_error& re )
        {
            std::cerr << re.what () << "\n";
          }
        catch(std::exception &e)
        {
              std::cerr << "Exception: " << e.what() << "\n";
          }
        catch(...)
        {
                std::cerr << "An unexpected exception was thrown. Caught in main.\n";
          }
        return 0;
      }
  \end{lstlisting}
 